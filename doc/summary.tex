
\documentclass[a4paper,oneside]{report}
%\documentclass[a4paper]{ctexart}

\usepackage{amssymb}
\usepackage{amsmath}
\usepackage{CJKutf8}
\usepackage{color}
\usepackage{enumerate}
\usepackage{fancyhdr}
\usepackage{geometry}
\usepackage{graphicx}
\usepackage{indentfirst}
\usepackage{latexsym}
\usepackage{mathrsfs}
\usepackage{subfigure}
\usepackage{textcomp}
\usepackage{url}
%\usepackage{unicode-math}

\usepackage{flafter}
\usepackage{booktabs, longtable}
\usepackage{pxfonts}
\usepackage{cite}

\DeclareMathAlphabet{\mathcal}{OMS}{cmsy}{m}{n}
\let\mathbb\relax % remove the definition by unicode-math
\DeclareMathAlphabet{\mathbb}{U}{msb}{m}{n}

\begin{document}
%\begin{CJK*}{UTF8}{gbsn}
\begin{CJK*}{UTF8}{gkai}
\CJKindent
%------------中文设置--------------------------
\makeatletter %将文献引用作为上标出现,增加括号,
\def\@cite#1#2{\textsuperscript{[{#1\if@tempswa , #2\fi}]}}
\makeatother
\newtheorem{theorem}{{定理}}
\newtheorem{remark}{{注}}
\newtheorem{proposition}[theorem]{{命题}}
\newtheorem{lemma}[theorem]{{引理}}
\newtheorem{corollary}[theorem]{{推论}}
\newtheorem{definition}[theorem]{{定义}}
\newtheorem{question}[theorem]{{问题}}

%\renewcommand{\refname}{\centerline{参考文献}}
\renewcommand{\bibname}{\centerline{参考文献}}
\renewcommand{\tablename}{表}
%\renewcommand{\captionlabeldelim}{\quad}
%===================Image settings========================%
\renewcommand{\figurename}{图}
%\renewcommand{\abstractname}{摘要}
%\renewcommand{\captionlabeldelim}{\quad} %Need caption2 macro package
%===============End image settings========================%
%-----------中文设置--------------------------

%%%%%%%%%%%
%% Change this!
%%%%%%%%%%%

\title{2022-2023程序/项目说明文档 \\ 2022-2023 programs\&projects documentation}
\author{杜政彤 \\ 浙江大学数学科学学院计算数学系 \\ zhengtongdu@zju.edu.cn}
\date{17 Nov, 2022}

\maketitle


\begin{center}
  \large
  序言
\end{center}

好的学习工具和思想方法可以让学习变的更加高效。
本书中总结的内容覆盖了张庆海老师研究团队的研究生必须掌握的基本工具和科
研方法。
第1章主要介绍一些科学计算中的常用工具,
从新系统的安装到其上一些软件(emacs,texlive,make,gdb 等等)的使用。
熟练的运用这些软件,往往可以达到事半功倍的效果;
第2章简单的介绍一些重要的思想及良好的习惯。
在以后的学习生涯中,养成这些习惯,贯彻这些思想,并融汇贯通,
将使你的科研大有裨益。


%%% Local Variables:
%%% mode: latex
%%% TeX-master: "../Guide"
%%% End:


\pagestyle{empty}

\tableofcontents
\clearpage

\pagestyle{fancy}
\fancyhead{}

%%%%%%%%%%%
%% Change this!
%%%%%%%%%%%
\lhead{zhengtong}
\chead{}
\rhead{program&projects documentation}

\chapter{基于分治法设计的二维点集计算凸包 \\(Computing the set of 2D points' Convex Hull based on Divide and Conquer)}\label{sec:convexHull}
%% 凸包设计文档

\section{问题描述}
给定二维平面上包含$n$个点的点集
\[S = \{(x_i, y_i)\ |\ i = 1,2,\cdots n\}\]
为了方便后面的设计,这里我们假设$x_i$互不相同,$y_i$互不相同,任意三个点不在同一条直线上;我们称$S$的凸包 (ConvexHull) $CH(S)$ 为包含 $S$ 的所有点集的所有凸多边形的交集,可以证明凸包的所有顶点都凸包是$S$上的点,因此 $CH(S)$ 可以用双向链表按顺时针方向记录凸包顶点的方式存储。

\section{设计思路}

解决这个问题采取的范式是“分而治之”的方法,整体算法可以分为如下几步:
\begin{enumerate}
\item 对点集中的所有点,按$x$坐标大小进行排序;
\item 递归求解$CHS = CH(S)$:
  \begin{enumerate}
  \item 若$\sharp S == 1 \rightarrow CHS = S$;
  \item 将$S$按$x$坐标大小分成相同大小的两个子集$L$和$R$;
  \item 分别计算$CHL = CH(L)$和$CHR = CH(R)$;
  \item $CHS = Merge(CHL, CHR)$。
  \end{enumerate}
\end{enumerate}


对于其中的$Merge(CHL, CHR)$,可以用$O(N)$的时间来完成,具体形式如下:
%\begin{enumerate}

%\end{enumerate}

对于二维平面上的点,我用point2.h这个头文件记录了point2这个模板类,它满足一般的赋值、运算操作。特别值得注意的是,对于比较操作,因为采用的方法中需要对点的$x$坐标进行排序,因此比较运算重写为对于$x$坐标进行比较。

\section{代码实现}

\section{算例测试}


\end{CJK*}
\end{document}

