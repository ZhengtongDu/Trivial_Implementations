%% 五则运算计算器设计文档

\section{问题描述}
给定一个包含加、减、乘、除和乘方运算的五则运算表达式字符串,根据运算律计算正确的结果。具体要求有如下三条:

\begin{enumerate}
\item 支持求解任意长度的含加、减、乘、除、乘方运算的表达式结果;
\item 允许浮点数参与运算,以及允许带括号的计算;
\item 能够对错误格式的表达式输入(包括除数为0、逻辑错误、错误字符等)给出响应,并提供相应的错误信息。
\end{enumerate}

\section{设计思路}

本项目主要利用的数据结构是基于二叉树的\textbf{表达式树},它的非叶节点为运算符或运算数,它的叶节点全部为操作数。如果能够根据某种规则构造该表达式树,那么根据只需要进行后序遍历便可以得到结果。

所以本次设计其实是“知果推因”,即在先有了表达式树的构造之后,再回过头思考如何生成期望格式的表达式树。

%% 这里打算插几张图,解释什么样的树是表达式树,怎么把算式转化为对应的表达式树,怎么去构造

从上面的例子,我们可以看到表达式树映射到算式的运算过程包含两个特点:第一,越是计算优先级高的运算符,在表达式树中的位置也就越深;第二,运算符在表达式树中的顺序和在算式中的顺序相同(即算式中越靠后的算式在表达式树中也越靠后)。因此我们可以通过输入的运算表达式,按从左到右的顺序构造相应的表达式树。

\section{代码实现}

\section{算例测试}
